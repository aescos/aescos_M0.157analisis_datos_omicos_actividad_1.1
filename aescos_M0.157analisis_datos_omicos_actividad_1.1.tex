% Options for packages loaded elsewhere
\PassOptionsToPackage{unicode}{hyperref}
\PassOptionsToPackage{hyphens}{url}
%
\documentclass[
]{article}
\usepackage{amsmath,amssymb}
\usepackage{iftex}
\ifPDFTeX
  \usepackage[T1]{fontenc}
  \usepackage[utf8]{inputenc}
  \usepackage{textcomp} % provide euro and other symbols
\else % if luatex or xetex
  \usepackage{unicode-math} % this also loads fontspec
  \defaultfontfeatures{Scale=MatchLowercase}
  \defaultfontfeatures[\rmfamily]{Ligatures=TeX,Scale=1}
\fi
\usepackage{lmodern}
\ifPDFTeX\else
  % xetex/luatex font selection
\fi
% Use upquote if available, for straight quotes in verbatim environments
\IfFileExists{upquote.sty}{\usepackage{upquote}}{}
\IfFileExists{microtype.sty}{% use microtype if available
  \usepackage[]{microtype}
  \UseMicrotypeSet[protrusion]{basicmath} % disable protrusion for tt fonts
}{}
\makeatletter
\@ifundefined{KOMAClassName}{% if non-KOMA class
  \IfFileExists{parskip.sty}{%
    \usepackage{parskip}
  }{% else
    \setlength{\parindent}{0pt}
    \setlength{\parskip}{6pt plus 2pt minus 1pt}}
}{% if KOMA class
  \KOMAoptions{parskip=half}}
\makeatother
\usepackage{xcolor}
\usepackage[margin=1in]{geometry}
\usepackage{color}
\usepackage{fancyvrb}
\newcommand{\VerbBar}{|}
\newcommand{\VERB}{\Verb[commandchars=\\\{\}]}
\DefineVerbatimEnvironment{Highlighting}{Verbatim}{commandchars=\\\{\}}
% Add ',fontsize=\small' for more characters per line
\usepackage{framed}
\definecolor{shadecolor}{RGB}{248,248,248}
\newenvironment{Shaded}{\begin{snugshade}}{\end{snugshade}}
\newcommand{\AlertTok}[1]{\textcolor[rgb]{0.94,0.16,0.16}{#1}}
\newcommand{\AnnotationTok}[1]{\textcolor[rgb]{0.56,0.35,0.01}{\textbf{\textit{#1}}}}
\newcommand{\AttributeTok}[1]{\textcolor[rgb]{0.13,0.29,0.53}{#1}}
\newcommand{\BaseNTok}[1]{\textcolor[rgb]{0.00,0.00,0.81}{#1}}
\newcommand{\BuiltInTok}[1]{#1}
\newcommand{\CharTok}[1]{\textcolor[rgb]{0.31,0.60,0.02}{#1}}
\newcommand{\CommentTok}[1]{\textcolor[rgb]{0.56,0.35,0.01}{\textit{#1}}}
\newcommand{\CommentVarTok}[1]{\textcolor[rgb]{0.56,0.35,0.01}{\textbf{\textit{#1}}}}
\newcommand{\ConstantTok}[1]{\textcolor[rgb]{0.56,0.35,0.01}{#1}}
\newcommand{\ControlFlowTok}[1]{\textcolor[rgb]{0.13,0.29,0.53}{\textbf{#1}}}
\newcommand{\DataTypeTok}[1]{\textcolor[rgb]{0.13,0.29,0.53}{#1}}
\newcommand{\DecValTok}[1]{\textcolor[rgb]{0.00,0.00,0.81}{#1}}
\newcommand{\DocumentationTok}[1]{\textcolor[rgb]{0.56,0.35,0.01}{\textbf{\textit{#1}}}}
\newcommand{\ErrorTok}[1]{\textcolor[rgb]{0.64,0.00,0.00}{\textbf{#1}}}
\newcommand{\ExtensionTok}[1]{#1}
\newcommand{\FloatTok}[1]{\textcolor[rgb]{0.00,0.00,0.81}{#1}}
\newcommand{\FunctionTok}[1]{\textcolor[rgb]{0.13,0.29,0.53}{\textbf{#1}}}
\newcommand{\ImportTok}[1]{#1}
\newcommand{\InformationTok}[1]{\textcolor[rgb]{0.56,0.35,0.01}{\textbf{\textit{#1}}}}
\newcommand{\KeywordTok}[1]{\textcolor[rgb]{0.13,0.29,0.53}{\textbf{#1}}}
\newcommand{\NormalTok}[1]{#1}
\newcommand{\OperatorTok}[1]{\textcolor[rgb]{0.81,0.36,0.00}{\textbf{#1}}}
\newcommand{\OtherTok}[1]{\textcolor[rgb]{0.56,0.35,0.01}{#1}}
\newcommand{\PreprocessorTok}[1]{\textcolor[rgb]{0.56,0.35,0.01}{\textit{#1}}}
\newcommand{\RegionMarkerTok}[1]{#1}
\newcommand{\SpecialCharTok}[1]{\textcolor[rgb]{0.81,0.36,0.00}{\textbf{#1}}}
\newcommand{\SpecialStringTok}[1]{\textcolor[rgb]{0.31,0.60,0.02}{#1}}
\newcommand{\StringTok}[1]{\textcolor[rgb]{0.31,0.60,0.02}{#1}}
\newcommand{\VariableTok}[1]{\textcolor[rgb]{0.00,0.00,0.00}{#1}}
\newcommand{\VerbatimStringTok}[1]{\textcolor[rgb]{0.31,0.60,0.02}{#1}}
\newcommand{\WarningTok}[1]{\textcolor[rgb]{0.56,0.35,0.01}{\textbf{\textit{#1}}}}
\usepackage{graphicx}
\makeatletter
\def\maxwidth{\ifdim\Gin@nat@width>\linewidth\linewidth\else\Gin@nat@width\fi}
\def\maxheight{\ifdim\Gin@nat@height>\textheight\textheight\else\Gin@nat@height\fi}
\makeatother
% Scale images if necessary, so that they will not overflow the page
% margins by default, and it is still possible to overwrite the defaults
% using explicit options in \includegraphics[width, height, ...]{}
\setkeys{Gin}{width=\maxwidth,height=\maxheight,keepaspectratio}
% Set default figure placement to htbp
\makeatletter
\def\fps@figure{htbp}
\makeatother
\setlength{\emergencystretch}{3em} % prevent overfull lines
\providecommand{\tightlist}{%
  \setlength{\itemsep}{0pt}\setlength{\parskip}{0pt}}
\setcounter{secnumdepth}{-\maxdimen} % remove section numbering
\ifLuaTeX
  \usepackage{selnolig}  % disable illegal ligatures
\fi
\usepackage{bookmark}
\IfFileExists{xurl.sty}{\usepackage{xurl}}{} % add URL line breaks if available
\urlstyle{same}
\hypersetup{
  pdftitle={aescos\_M0.157analisis\_datos\_omicos\_actividad\_1.1},
  hidelinks,
  pdfcreator={LaTeX via pandoc}}

\title{aescos\_M0.157analisis\_datos\_omicos\_actividad\_1.1}
\author{}
\date{\vspace{-2.5em}2024-10-02}

\begin{document}
\maketitle

\begin{Shaded}
\begin{Highlighting}[]
\FunctionTok{rm}\NormalTok{(}\AttributeTok{list =} \FunctionTok{ls}\NormalTok{())}
\end{Highlighting}
\end{Shaded}

\subsubsection{1. Select a GEO (Gene Expression Omnibus) datasetfrom the
list presented in the ``GEOdatasets\_enhanced.xls'' document available
in the resources of the
activity.}\label{select-a-geo-gene-expression-omnibus-datasetfrom-the-list-presented-in-the-geodatasets_enhanced.xls-document-available-in-the-resources-of-the-activity.}

He seleccionado el data set MyD88 deficient macrophage response to
zymosan. Estudio realizado en mus musculus (ratón) dataset: GDS2686,
Serie: GSE6376

\subsubsection{2. Leer los datos desde GEO utilizando el paquete
GEOquery. Esto os proporcionará un objeto clase expressionSet con los
datos normalizados y una tabla adicional con información sobre el
estudio.}\label{leer-los-datos-desde-geo-utilizando-el-paquete-geoquery.-esto-os-proporcionaruxe1-un-objeto-clase-expressionset-con-los-datos-normalizados-y-una-tabla-adicional-con-informaciuxf3n-sobre-el-estudio.}

\begin{Shaded}
\begin{Highlighting}[]
\FunctionTok{require}\NormalTok{(GEOquery)}
\end{Highlighting}
\end{Shaded}

\begin{verbatim}
## Loading required package: GEOquery
\end{verbatim}

\begin{verbatim}
## Loading required package: Biobase
\end{verbatim}

\begin{verbatim}
## Loading required package: BiocGenerics
\end{verbatim}

\begin{verbatim}
## 
## Attaching package: 'BiocGenerics'
\end{verbatim}

\begin{verbatim}
## The following objects are masked from 'package:stats':
## 
##     IQR, mad, sd, var, xtabs
\end{verbatim}

\begin{verbatim}
## The following objects are masked from 'package:base':
## 
##     anyDuplicated, aperm, append, as.data.frame, basename, cbind,
##     colnames, dirname, do.call, duplicated, eval, evalq, Filter, Find,
##     get, grep, grepl, intersect, is.unsorted, lapply, Map, mapply,
##     match, mget, order, paste, pmax, pmax.int, pmin, pmin.int,
##     Position, rank, rbind, Reduce, rownames, sapply, setdiff, table,
##     tapply, union, unique, unsplit, which.max, which.min
\end{verbatim}

\begin{verbatim}
## Welcome to Bioconductor
## 
##     Vignettes contain introductory material; view with
##     'browseVignettes()'. To cite Bioconductor, see
##     'citation("Biobase")', and for packages 'citation("pkgname")'.
\end{verbatim}

\begin{verbatim}
## Setting options('download.file.method.GEOquery'='auto')
\end{verbatim}

\begin{verbatim}
## Setting options('GEOquery.inmemory.gpl'=FALSE)
\end{verbatim}

\begin{Shaded}
\begin{Highlighting}[]
\NormalTok{gds }\OtherTok{\textless{}{-}} \FunctionTok{getGEO}\NormalTok{(}\StringTok{"GDS2686"}\NormalTok{)}
\end{Highlighting}
\end{Shaded}

\subsubsection{3. Determinar la estructura de los datos (filas,
columnas)y el diseño del estudio (grupos de muestras o individuos,
tratamientos si los hay, etc.) que los ha
generado.}\label{determinar-la-estructura-de-los-datos-filas-columnasy-el-diseuxf1o-del-estudio-grupos-de-muestras-o-individuos-tratamientos-si-los-hay-etc.-que-los-ha-generado.}

-- La información del experimento podéis descargarla también de GEO,
bien con GEOquery si proporcionáis el identificador de dataset GDSxxxx o
accediendo a la página del estudio.

\begin{Shaded}
\begin{Highlighting}[]
\FunctionTok{class}\NormalTok{(gds)}
\end{Highlighting}
\end{Shaded}

\begin{verbatim}
## [1] "GDS"
## attr(,"package")
## [1] "GEOquery"
\end{verbatim}

\begin{Shaded}
\begin{Highlighting}[]
\FunctionTok{slotNames}\NormalTok{(gds)}
\end{Highlighting}
\end{Shaded}

\begin{verbatim}
## [1] "gpl"       "dataTable" "header"
\end{verbatim}

\begin{Shaded}
\begin{Highlighting}[]
\FunctionTok{head}\NormalTok{(}\FunctionTok{Meta}\NormalTok{(gds))}
\end{Highlighting}
\end{Shaded}

\begin{verbatim}
## $channel_count
## [1] "1"
## 
## $dataset_id
## [1] "GDS2686" "GDS2686"
## 
## $description
## [1] "Analysis of MyD88-deficient macrophages treated with fungal cell wall component zymosan. Zymosan-induced Toll-like receptor (TLR) signaling is transduced via adaptor protein MyD88. Results provide insight into the role of non-TLR innate immune signaling in response to zymosan and pathogenic fungi."
## [2] "control"                                                                                                                                                                                                                                                                                                   
## [3] "zymosan"                                                                                                                                                                                                                                                                                                   
## 
## $email
## [1] "geo@ncbi.nlm.nih.gov"
## 
## $feature_count
## [1] "45101"
## 
## $institute
## [1] "NCBI NLM NIH"
\end{verbatim}

\end{document}
